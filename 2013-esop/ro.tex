\subsection{Read-only permissions}

We formalise the read-only counting permissions we used in
Section~\ref{sec:hash} as the following separation algebra $\RAM_{\sf ro}$:
$$
\Loc = \{1, 2, \ldots\};
\qquad\quad
\Val = \mathbb{Z};
\qquad\quad
\RAM_{\sf ro} = \Loc \rightharpoonup_{\it fin} (\mathbb{Z} \times \Val).
$$
Like in Section~\ref{sec:exist}, the permission $0$ is the {\em full}
permission, allowing a thread to read, write and deallocate the corresponding
cell. Such a permission can be split into several {\em read-only} permissions
and one {\em master} permission, keeping track of the number of read-only ones
issued. Both read-only and master permissions guarantee that the cell is not
going to be modified and give the right to read from it, but not write to it or
dispose it. One read-only permission is represented in the model by $-1$, and
several of them by negative integers; a master permission is represented by a
positive integer.

We first define a $*$ operation on permission-value
pairs from $\mathbb{Z} \times \Val$:
$$
\begin{array}{c@{\quad}l}
(-\pi_1, u) * (-\pi_2, u) = (-(\pi_1+\pi_2), u), & \mbox{if } \pi_1,\pi_2>0;
\\
(\pi_1, u) * (-\pi_2, u) = (-\pi_2, u) * (\pi_1, u) = (\pi_1-\pi_2, u),&
\mbox{if } \pi_1,\pi_2>0, \pi_1 \ge \pi_2;
\end{array}
$$
undefined in all other cases. We then define the $\sharp$ relation on
$\RAM_{\sf ro}$ by
$$
\theta_0 \mathop{\sharp} \theta_1 \Longleftrightarrow
\forall i \in \{0, 1\}.\, {\theta_i(x) \fdef} \Rightarrow 
{\theta_{1-i}(x) \fundef} \vee {(\theta_i(x) * \theta_{1-i}(x)) \fdef}.
$$
Finally, we let
\begin{multline*}
\theta_0 * \theta_1 = \{(x, w) \mid 
\exists i \in \{0, 1\}.\, {\theta_i(x) \fdef} \wedge{}\\
(({\theta_{1-i}(x) \fundef} \wedge w = \theta_i(x)) \vee 
({\theta_{1-i}(x) \fdef} \wedge w = \theta_i(x) * \theta_{1-i}(x)))
\},
\end{multline*}
if $\theta_0 \mathrel{\sharp} \theta_1$; undefined otherwise. Thus, the unit
element is the nowhere-defined function $[\,]$.

To denote elements of $\RAM_{\sf ro}$, we extend the assertion language for
predicates over states as follows:
$
p \;::= \; \ldots \mid E \xmapsto{G}_{\sf ro} F \mid E \xhookrightarrow{G}_{\sf ro} \_ 
$. Here $E \xmapsto{0}_{\sf ro} F$ denotes a full permission, 
$E \xmapsto{G}_{\sf ro} F$ for $G>0$, a master permission, and 
$E \xhookrightarrow{G}_{\sf ro} \_ $, read-only ones:
$$
\begin{array}{lcl@{\qquad}l}
 (\theta, \intp) \models E \xmapsto{G}_{\sf ro} F &\Longleftrightarrow &\theta =
[\db{E}_{\intp} : (\db{G}_\intp, \db{F}_{\intp})],&\db{G}_\intp \ge 0;
\\ 
(\theta, \intp) \models E \xhookrightarrow{G}_{\sf ro} F 
&\Longleftrightarrow & 
\theta = [\db{E}_{\intp} : (-\db{G}_\intp, \db{F}_{\intp})],&\db{G}_\intp > 0;\\
(\theta, \intp) \models E \xhookrightarrow{0}_{\sf ro} F &\Longleftrightarrow
& \theta = [\,].
\end{array}
$$
% We also use an iterated $*$ operation: $\bigast_{i=1}^n P_i = P_1 * \ldots *
% P_n$, so that $\bigast_{i=1}^n E \hookrightarrow_{\sf e} \_$ represents $n$ existential
% permissions for the cell at the address $E$.
As in Section~\ref{sec:exist}, we use $E \mapsto F$ as syntactic sugar for $E
\xmapsto{0}_{\sf ro} F$ and $E \hookrightarrow_{\sf ro} F$ for $E
\xhookrightarrow{1}_{\sf ro} F$.  

% Given the definition of $*$ on $\RAM_{\sf ro}$, we have the following
% equivalences:
% $$
% \begin{array}{c}
% E \xmapsto{0}_{\sf ro} F \Longleftrightarrow 
% E \xmapsto{n}_{\sf ro} F * E \xhookrightarrow{n}_{\sf ro} F; \quad
% E \xhookrightarrow{n}_{\sf ro} F * E \xhookrightarrow{n'}_{\sf ro} F
% \Longleftrightarrow 
% E \xhookrightarrow{n+n'}_{\sf ro} F;
% \\[5pt]
% E \xmapsto{0}_{\sf ro} F * E \xmapsto{0}_{\sf ro} F \Longleftrightarrow {\sf
%   false};
% \qquad
% E \xmapsto{0}_{\sf ro} F * E \xhookrightarrow{n}_{\sf ro} F \Longleftrightarrow {\sf false};
% \\[5pt]
% E \xhookrightarrow{n}_{\sf ro} F * E \xhookrightarrow{n'}_{\sf ro} F' \wedge
% F \not= F'
% \Longleftrightarrow 
% {\sf false};
% \end{array}
% $$
% where $n, n'> 0$. 

Transformers and axioms for primitive commands over $\RAM_{\sf ro}$ are defined
as in Section~\ref{sec:exist2}, but using the relation $\leadsto$ in
Figure~\ref{fig:transfer2}. Figure~\ref{fig:prim2} shows sample axioms for
primitive commands.
\begin{figure}[h]
\centerline{
$
\begin{array}{l@{\ \ }l@{\quad}l@{\quad}l@{\quad}l}
[X] = Y, &\theta[X : (0, \_)] &\leadsto& \theta[X: (0,Y)]
\\{}
[Y] = [X], &\theta[X : (\pi, u)][Y : (0, \_)] &\leadsto&
\theta[X:(\pi, u)][Y:(0, u)]
\\
\new([X]), &\theta[X : (0, \_)] &\leadsto&
\theta[X: (0, u)][u : (0, \_)]& \text{if}~{(\theta[X : \_])(u)\fundef}
\\
\delete([X]),&\theta[X : (0, u)][u : (0, \_)] &\leadsto&
\theta[X:(0,u)]& \text{if}~{(\theta[X:\_])(u)\fundef}
\\
{\sf assume}([X]), &\theta[X : (\pi, u)] &\leadsto& \theta[X : (\pi, u)]
&\text{if}~u\not=0
\\
{\sf assume}([X]), &\theta[X : (\pi, u)] &\not\leadsto&
&\text{if}~u =0
\\
c, &\theta &\leadsto& \top
&\text{otherwise}
\end{array}
$}
\caption{\label{fig:transfer2}
Transition relation for primitive commands over $\RAM_{\sf ro}$.
$\top$ indicates that the command faults.
$\not\leadsto$ is used to denote that the command does not fault, but gets stuck.
}
\end{figure}
\begin{figure}[h]
$$
\begin{array}{c}
\\
\infer%[\textsc{Store}]
{\{X \mapsto \_\}\, [X] = Y\, \{X \mapsto Y\}}{}
\\[10pt]
\infer%[\textsc{Load}_1]
{\{X \xmapsto{n}_{\sf ro} u*
Y \mapsto \_ \}\, [Y] = [X]\, 
\{X \xmapsto{n}_{\sf ro} u *
Y \mapsto u \}}{}
\\[10pt]
\infer%[\textsc{Load}_2]
{\{X \hookrightarrow_{\sf ro} \_*
Y \mapsto \_ \}\, 
[Y] = [X]\, 
\{X \hookrightarrow_{\sf ro} \_*
Y \mapsto \_ \}}{}
\\[10pt]
\infer%[\textsc{New}]
{\{X \mapsto \_ \}\, \new(X)\, \{\exists u.\, X \mapsto u*
u \mapsto \_\}}{}
\\[10pt]
\infer%[\textsc{Delete}]
{\{X \xmapsto{n}_{\sf e} u * u \mapsto \_\}\, \delete(X)\, \{
X \xmapsto{n}_{\sf e} u\}}{}
\\[10pt]
\infer%[\textsc{Delete}]
{\{X \hookrightarrow_{\sf ro} u * u \mapsto \_\}\, \delete(X)\, \{
X \hookrightarrow_{\sf ro} u\}}{}
\\[10pt] 
\infer
{\{X \xmapsto{n}_{\sf ro} u\}\, {\sf assume}([X])\, \{X \xmapsto{n}_{\sf ro} u
  \wedge u \not= 0\}}{}
\\[10pt] 
\infer
{\{X \hookrightarrow_{\sf ro} u\}\, {\sf assume}([X])\, \{X \hookrightarrow_{\sf ro} u
  \wedge u \not= 0\}}{}
\end{array}
$$
\caption{Instantiations of the {\sc Prim} axiom for the $\RAM_{\sf ro}$
  algebra; $X$ and $Y$ are constants}
\label{fig:prim2}
\end{figure}

%%% Local Variables:
%%% TeX-master: "rcu"
%%% End:  
