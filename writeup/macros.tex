%!TEX root=./main.tex

\newcommand{\ignore}[1]{}
\newcommand{\TBD}{\framebox{\textbf{TBD}}}

\newcommand{\nrtodo}[1]{\color{red}\textbf{[NR:~#1]}\color{black}}
\newcommand{\noam}[1]{\todo[inline]{NR: #1}}
\newcommand{\red}[1]{\color{red}#1\color{black}}
\newcommand{\blue}[1]{\color{blue}#1\color{black}}
\newcommand{\purple}[1]{\color{purple}#1\color{black}}
\newcommand{\code}[1]{\texttt{#1}}
\newcommand{\semfunc}[1][]{\llbracket \texttt{#1} \rrbracket}
\newcommand{\emptymap}[0]{()}
\newcommand{\tbd}[1]{}

%%%%%%%%%%%%%%%%%%%%%%%%%%%%%%%%%%%%%%%%%%%%%%%%%%%%%%%%%%%
%%%%%%%%%%%%%%%%%%%%%%%%%%%%  Math  %%%%%%%%%%%%%%%%%%%%%%%
%%%%%%%%%%%%%%%%%%%%%%%%%%%%%%%%%%%%%%%%%%%%%%%%%%%%%%%%%%%

\makeatletter
\newcommand{\@@eqLikeDef}[3]{%
    \ensuremath{\overset{\mathclap{\text{\scalebox{#1}{#2}}}}{#3}}%
}
\newcommand*{\@eqLikeDef}[2]{
    \mathchoice
        {% \displaystyle
            \@@eqLikeDef{0.7}{#1}{#2}
        }
        {% \textstyle
            \@@eqLikeDef{0.7}{#1}{#2}
        }
        {% \scriptstyle
            \@@eqLikeDef{0.6}{#1}{#2}
        }
        {% \scriptscriptstyle
            \@@eqLikeDef{0.5}{#1}{#2}
        }
}

\newcommand*{\eqdef}{\@eqLikeDef{def}{=}}
\makeatother


%%%%%%%%%%%%%%%%%%%%%%%%%%%%%%%%%%%%%%%%%%%%%%%%%%%%%%%%%%%

\newcommand{\PDA}{\text{PDA}}
% Standard macros
\newcommand{\B}[1]{\langle #1 \rangle}
\newcommand{\ST}[1]{(#1)}
\newcommand{\SET}[1]{\{#1\}}
\newcommand{\partialto}{\rightharpoonup}
\newcommand{\totalto}{\to}
\newcommand{\undefunc}{\bot}
\newcommand{\dom}{\operatorname{dom}}
\newcommand{\range}{\operatorname{range}}
\newcommand{\goesto}[1][]{\stackrel{#1}{\rightsquigarrow}}
\newcommand{\neggoesto}[1][]{\stackrel{#1}{\nrightsquigarrow}}

\newcommand{\res}{\mathit{res}}
\newcommand{\ret}{\code{ret}}
\newcommand{\inarg}{\code{arg}}
%\newcommand{\ret}{\mathit{ret}}
\newcommand{\formal}[1]{\operatorname{formal}(o)}



\newcommand{\PCmd}{\syn{PCmd}}
\newcommand{\Cmd}{\syn{Cmd}}
\newcommand{\Contract}{\syn{Contract}}
\newcommand{\Mthd}{\syn{Mthd}}

% Instructions
\newcommand{\SAssign}[2]{#1 \coloneqq  #2}
\newcommand{\SFault}{\code{fault}}
\newcommand{\SAssume}[1]{\code{assume}(#1)}
\newcommand{\SAssert}[1]{\code{assert}(#1)}
\newcommand{\SSkip}[0]{\code{skip}}
\newcommand{\SSLoad}[2]{\code{#1 := S[#2]}}
\newcommand{\SSStore}[2]{\code{S[#1] := #2}}
\newcommand{\SMLoad}[2]{\code{#1 := M[#2]}}
\newcommand{\SMStore}[2]{\code{M[#1] := #2}}
\newcommand{\SCall}[4]{{#1 \coloneqq #2.#3 (#4)}}
\newcommand{\SCallC}[3]{{#1 \coloneqq #2(#3)}}
\newcommand{\SCallCO}[2]{{#1(#2)}}
\newcommand{\SReturn}[0]{\code{return}}
\newcommand{\SReturnV}{\code{return}}
% \newcommand{\SEnterV}[1]{\code{enter(#1)}}
\newcommand{\SEnterV}{\code{enter}}%\,(\inarg)}}
\newcommand{\SMethodDecl}[4]{{\Method \,\, #1(#2) : #3 \,\, #4}}
\newcommand{\cdone}{\code{done}}

% Map instructions
\newcommand{\SInsert}[3]{\code{insert(#1,#2,#3)}}
\newcommand{\SGet}[2]{\code{get(#1,#2)}}
\newcommand{\SRemove}[2]{\code{remove(#1,#2)}}


\newcommand{\PLname}{\ensuremath{\mathbf{SMAC}}}

% % Syntax
% \newcommand{\ObjectDecl}[0]{\code{ObjectDecl}}
% \newcommand{\RecordDef}[4]{\Record \,\, #1 \,\, \{ \Int \,\, #2, \Bool \,\, #3, \Object \,\, #4\}}
% \newcommand{\RecordDecl}[0]{\code{RecordDecl}}
% \newcommand{\FieldDecl}[0]{\code{FieldDecl}}
% \newcommand{\FieldDef}[2]{#1 \,\, \field{#2}}
% \newcommand{\MethodDecl}[0]{\code{MethodDecl}}
% \newcommand{\MethodDef}[5]{\SMethodDecl{#1}{#2}{#3}{#4}\{ #5 \}}

% % Client syntax
% \newcommand{\CSExecute}[3]{\code{Execute #1,#2,#3}}
% \newcommand{\CSTransfer}[2]{\code{Transfer #1,#2}}

% \newcommand{\CSNew}[5]{\code{#5 = new}\{#1 #2 ; #3 ; #4\}}
% \newcommand{\CSInvoke}[3]{#1.#2(#3)}
% \newcommand{\defaultCSNew}[0]{\CSNew{\overline{\code{T}}}{\overline{\,\,\field{f}}}{\method{init}}{\overline{\method{m}}}{o}}
% \newcommand{\defaultCSInvoke}[0]{\CSInvoke{\var{o}}{\method{m}}{\overline{\code{constant}}}}

% \newcommand{\CSNewO}[5]{\code{new}(\B{#1 \,#2 \,; \,\,#3\,;\,\, #4}, #5)}
% \newcommand{\defaultCSNewO}[0]{\CSNewO{\overline{\code{T}}}{\overline{\,\,\field{f}}}{\method{init}}{\overline{\method{m}}}{o}}


% Exprssions
% \newcommand{\EVar}[1]{\code{#1}}

% Syntactic domains
\newcommand{\syn}[1]{\mathsf{#1}}
\newcommand{\Var}[0]{\syn{LVar}}
\newcommand{\Inst}[0]{\code{Inst}}
% \newcommand{\Action}[0]{\code{Action}}
\newcommand{\Field}[0]{\syn{OVar}}
\newcommand{\Int}[0]{\code{Int}}
\newcommand{\Bool}[0]{\code{Bool}}
\newcommand{\Object}{\sem{Obj}}
% \newcommand{\Record}[0]{\code{Record}}
% \newcommand{\Map}[2]{\code{Map<#1,#2>}}
% \newcommand{\Bexp}[0]{\code{Bexp}}
% \newcommand{\Aexp}[0]{\code{Aexp}}
% \newcommand{\Method}[0]{\code{Method}}
\newcommand{\field}[1]{\code{\uppercase{#1}}}
\newcommand{\var}[1]{\code{\lowercase{#1}}}
\newcommand{\this}[0]{\code{this}}
% \newcommand{\method}[1]{\code{\lowercase{#1}}}

% Semantic domains
\newcommand{\semtrue}[0]{\it{true}}
\newcommand{\semfalse}[0]{\it{false}}
% \newcommand{\Caddress}[0]{O}
% \newcommand{\PC}[0]{PC}
% \newcommand{\Loc}[0]{Loc}
% \newcommand{\Usr}[0]{Usr}
% \newcommand{\Prog}[0]{Prog}

% \newcommand{\World}[0]{\mathcal{W}}
% \newcommand{\ProgPoint}[0]{K}

% \newcommand{\SelStack}{\it{Stack}}
% \newcommand{\SelObject}{\it{This}}
% \newcommand{\SelCmd}{\it{Cmd}}
% \newcommand{\SelStore}{\it{Stor}}
% % \newcommand{\MutG}[0]{G}
% % \newcommand{\MutL}[0]{L}
% \newcommand{\SelEnv}{\it{Env}}
% \newcommand{\State}[0]{EState}
\newcommand{\SelStack}{\Stk}
\newcommand{\SelObject}{\objectid}
\newcommand{\SelCmd}{C}
\newcommand{\SelStore}{\store}
% \newcommand{\MutG}[0]{G}
% \newcommand{\MutL}[0]{L}
\newcommand{\SelEnv}{\reg}

\newcommand{\judge}[3] {\frac{#2}{#3} #1}
\newcommand{\reduce}[0] {\leadsto}


% Semantic Domains - new
% \newcommand{\CState}[0]{\Sigma}
% \newcommand{\OState}[0]{S}
% \newcommand{\Heap}[0]{\mathcal{H}}
\newcommand{\Reg}[0]{\mathcal{E}}
% \newcommand{\CFG}[0]{CFG}
% \newcommand{\Code}[0]{\mathcal{C}}
% \newcommand{\DomObject}[0]{\mathcal{O}}
% \newcommand{\PrimitiveValue}[0]{PrimitiveVal}
\newcommand{\sem}[1]{\mathbf{#1}}
\newcommand{\Val}{\sem{Val}}
\newcommand{\Stack}{\sem{Stack}}
\newcommand{\Store}{\sem{Store}}
\newcommand{\Frame}{\sem{Frame}}
\newcommand{\State}{\sem{State}}

\DeclareFontFamily{U} {MnSymbolA}{}
\DeclareFontShape{U}{MnSymbolA}{m}{n}{
  <-6> MnSymbolA5
  <6-7> MnSymbolA6
  <7-8> MnSymbolA7
  <8-9> MnSymbolA8
  <9-10> MnSymbolA9
  <10-12> MnSymbolA10
  <12-> MnSymbolA12}{}
\DeclareFontShape{U}{MnSymbolA}{b}{n}{
  <-6> MnSymbolA-Bold5
  <6-7> MnSymbolA-Bold6
  <7-8> MnSymbolA-Bold7
  <8-9> MnSymbolA-Bold8
  <9-10> MnSymbolA-Bold9
  <10-12> MnSymbolA-Bold10
  <12-> MnSymbolA-Bold12}{}

\DeclareFontFamily{U}  {MnSymbolB}{}
\DeclareFontShape{U}{MnSymbolB}{m}{n}{
    <-6>  MnSymbolB5
   <6-7>  MnSymbolB6
   <7-8>  MnSymbolB7
   <8-9>  MnSymbolB8
   <9-10> MnSymbolB9
  <10-12> MnSymbolB10
  <12->   MnSymbolB12}{}
\DeclareFontShape{U}{MnSymbolB}{b}{n}{
    <-6>  MnSymbolB-Bold5
   <6-7>  MnSymbolB-Bold6
   <7-8>  MnSymbolB-Bold7
   <8-9>  MnSymbolB-Bold8
   <9-10> MnSymbolB-Bold9
  <10-12> MnSymbolB-Bold10
  <12->   MnSymbolB-Bold12}{}

\DeclareSymbolFont{MnSyA} {U} {MnSymbolA}{m}{n}
\DeclareSymbolFont{MnSyB}         {U}  {MnSymbolB}{m}{n}

\DeclareMathSymbol{\rcirclearrowright}{\mathrel}{MnSyA}{248}
\DeclareMathSymbol{\nrcirclearrowright}{\mathrel}{MnSyB}{248}
\DeclareMathSymbol{\leftrightarrows}{\mathrel}{MnSyA}{152}
\DeclareMathSymbol{\nleftrightarrows}{\mathrel}{MnSyB}{152}

% \newcommand{\callback}[2][]{\lefttorightarrow{#2}}
\newcommand{\callback}[2][]{{#1}\!\rcirclearrowright{#2}}
\newcommand{\ncallback}[2][]{{#1}\!\nrcirclearrowright{#2}}
\newcommand{\callbackU}[1][]{\rcirclearrowright{\!#1}}
\newcommand{\ncallbackU}[1][]{\nrcirclearrowright{\!#1}}
\newcommand{\callbackfree}[1]{\ncallbackU{#1}}

% \newcommand{\callback}[2][]{\stackrel{\circlearrowright}{#2}}

% \newcommand{\callback}[2][]{\stackrel{\stackrel{#1}{\curvearrowright}}{#2}}
% \newcommand{\callback}[2][]{\operatorname{cb}_{#1}(#2)}
% \newcommand{\callback}[2][]{\stackrel{{\scriptscriptstyle\curvearrowright #1}}{#2}}

% \newcommand{\callback}[2][]{\stackrel{#1}{\widearc{#2}}}



% From Noam's
% \newcommand{\CStore}{\mathcal{S}}
% \newcommand{\OCode}{\mathcal{C}}
\newcommand{\ocode}{\kappa}
% \newcommand{\oheap}{\operatorname{heap}}
\newcommand{\ObjVar}{\Psi}
\newcommand{\oovar}{\operatorname{\psi}}

% Elements of semantic domains - new
% \newcommand{\cstate}[0]{\sigma}
% \newcommand{\ostate}[0]{s}
\newcommand{\objectid}{o}
% \newcommand{\pc}[0]{pc}
\newcommand{\cmd}[0]{c}
\newcommand{\seq}[1]{\overline{#1}}
\newcommand{\Stk}[0]{\Gamma}
\newcommand{\stk}[0]{\gamma}
\newcommand{\topstk}{\operatorname{top}}
\newcommand{\reg}[0]{\rho}
\newcommand{\defostatenostack}[4]{\langle \frame{#1}{#2}{#3}, #4 \rangle}
\newcommand{\defostatestack}[5]{\langle \frame{#1}{#2}{#3}\cdot #4, #5 \rangle}
\newcommand{\defostate}[4]{\defostatestack{#1}{#2}{#3}{\Stk}{#4}}
\newcommand{\ostatedefaultactive}{\defostate{\objectid}{\cmd}{\reg}{\store}}
\newcommand{\ostatedefault}{\langle \Stk,\store \rangle}
\newcommand{\errorstate}{\lightning}
\newcommand{\current}{\mathit{this}}
%\newcommand{\stack}{\mathit{Stack}}
\newcommand{\store}{\sigma}
\newcommand{\MCmd}{\operatorname{c}}
\newcommand{\Tr}{\mathit{Tr}}
\newcommand{\ETr}{\hat{\mathit{Tr}}}
\newcommand{\execdepth}{\it{Depth}}
\newcommand{\Run}{\it{Run}}
\newcommand{\Complete}{\it{Complete}}
\newcommand{\inv}{\it{I}}

% % Actions
% \newcommand{\ENTERshort}[0]{ENTER}
% \newcommand{\ENTER}[3]{ENTER\,\,#1.#2(#3)}
% \newcommand{\EXIT}[0]{EXIT}
% \newcommand{\THROW}[0]{THROW}

% Elements of semantic domains
% \newcommand{\defBstate}[2]{\langle #1, #2 \rangle}
% \newcommand{\defstate}[4]{\langle #1, #2, #3, #4 \rangle}
% \newcommand{\id}[0]{a}
% \newcommand{\usr}[0]{u}
% \newcommand{\prog}[0]{c}
% \newcommand{\heap}[0]{h}
% \newcommand{\mem}[0]{m}
% \newcommand{\bal}[0]{b}
% \newcommand{\stor}[0]{\psi}
\renewcommand{\state}{s}
\renewcommand{\frame}[3]{\ST{#1,\,#2,\,#3}}

% Rules of small-step semantics
\newcommand{\semsmallsteprule}[4]{\langle Code(#1), #2 \rangle \Rightarrow^{#3} #4}

% Rules of big-step semantics
\newcommand{\semrule}[3]{\frac{#1}{#2} & \textsc{#3} \\ \quad \\}

% Execution & Derivation trees
\newcommand{\src}{\operatorname{src}}
\newcommand{\trg}{\operatorname{trg}}
% \newcommand{\first}{\mathit{First}}
% \newcommand{\last}{\mathit{Last}}
\newcommand{\tree}[0]{\mathcal{T}}
\newcommand{\exec}[0]{\pi}
\newcommand{\Exec}[0]{\Pi}
\newcommand{\Leaves}[0]{Leaves}
\newcommand{\Conflict}{\it{Conflict}}
\newcommand{\CBE}[2]{#1\ensuremath{\prec_{\it{CBE}}}#2}
\newcommand{\CBEtc}[2]{#1\ensuremath{\prec^*_{\it{CBE}}}#2}
\newcommand{\cep}{\varphi}
\newcommand{\transition}{\tau}
\newcommand{\trid}{\iota}
\newcommand{\TRIDDOM}{I}
\newcommand{\event}{e}
\newcommand{\trace}{T}
\newcommand{\action}{a}


\newcommand{\TrID}[1]{\iota(#1)}
\newcommand{\subexec}{\sqsubseteq}
\newcommand{\projectedsubexec}[1]{\sqsubseteq_{#1}}
\newcommand{\proj}[1]{|_{#1}}
\newcommand{\fseq}[0]{\simeq_{FS}}
\newcommand{\ceq}[0]{\simeq_{C}}
\newcommand{\stkdeq}[2]{\ensuremath{\it{StkDeq(#1,#2)}}}
\newcommand{\cbunnest}[1]{\ensuremath{\it{CBUnnest(#1)}}}
\newcommand{\AllocObjects}[1]{\operatorname{Obj}(#1)}



% Theorems
\newcommand{\ECF}{\textsc{ECF}}
\newcommand{\SECF}[1][]{\textsc{staECF}#1}
\newcommand{\DECF}[1][]{\textsc{dynECF}#1}
\newcommand{\FSECF}{\textsc{ECF}\subFS}
\newcommand{\CECF}{\textsc{ECF}\subC}
\newcommand{\CF}{\textsc{CF}}
\newcommand{\subC}{\textsubscript{\textsc{c}}}
\newcommand{\subFS}{\textsubscript{\textsc{fs}}}

% \newcommand{\FSECF}{\ensuremath{\mathit{FS-ECF}}}
% \newcommand{\CECF}{\ensuremath{\mathit{C-ECF}}}
% \newcommand{\CF}{\ensuremath{\mathit{CF}}}


% Dynamic verification - instrumentation
\newcommand{\segment}[0]{t}
% \newcommand{\trace}[0]{\tau}
\newcommand{\exectree}{\mathcal{T}}
\newcommand{\forest}{\mathcal{F}}
\newcommand{\readset}[1]{R\ifthenelse{\equal{#1}{}}{}{(#1)}}
\newcommand{\writeset}[1]{W\ifthenelse{\equal{#1}{}}{}{(#1)}}
\newcommand{\depth}[1]{D\ifthenelse{\equal{#1}{}}{}{(#1)}}
\newcommand{\depthNoArg}{D}
\newcommand{\indexInCall}[1]{i\ifthenelse{\equal{#1}{}}{}{(#1)}}
\newcommand{\indexInExec}[1]{Idx\ifthenelse{\equal{#1}{}}{}{(#1)}}
\newcommand{\indexInExecNoArg}{Idx}
\newcommand{\prefix}[1]{\it{prefix}(#1)}
\newcommand{\suffix}[1]{\it{suffix}(#1)}
\newcommand{\Commute}[2]{#1\ensuremath{\leftrightarrows}#2}
\newcommand{\nCommute}[2]{#1\ensuremath{\nleftrightarrows}#2}
\newcommand{\hb}[2]{#1\ensuremath{\prec_{\it{Inv}}}#2} 