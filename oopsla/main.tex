%% For double-blind review submission, w/o CCS and ACM Reference (max submission space)
\documentclass[acmsmall,review,anonymous]{acmart}\settopmatter{printfolios=true,printccs=false,printacmref=false}
%% For double-blind review submission, w/ CCS and ACM Reference
%\documentclass[acmsmall,review,anonymous]{acmart}\settopmatter{printfolios=true}
%% For single-blind review submission, w/o CCS and ACM Reference (max submission space)
%\documentclass[acmsmall,review]{acmart}\settopmatter{printfolios=true,printccs=false,printacmref=false}
%% For single-blind review submission, w/ CCS and ACM Reference
%\documentclass[acmsmall,review]{acmart}\settopmatter{printfolios=true}
%% For final camera-ready submission, w/ required CCS and ACM Reference
%\documentclass[acmsmall]{acmart}\settopmatter{}


%% Journal information
%% Supplied to authors by publisher for camera-ready submission;
%% use defaults for review submission.
\acmJournal{PACMPL}
\acmVolume{1}
\acmNumber{CONF} % CONF = POPL or ICFP or OOPSLA
\acmArticle{1}
\acmYear{2018}
\acmMonth{1}
\acmDOI{} % \acmDOI{10.1145/nnnnnnn.nnnnnnn}
\startPage{1}

%% Copyright information
%% Supplied to authors (based on authors' rights management selection;
%% see authors.acm.org) by publisher for camera-ready submission;
%% use 'none' for review submission.
\setcopyright{none}
%\setcopyright{acmcopyright}
%\setcopyright{acmlicensed}
%\setcopyright{rightsretained}
%\copyrightyear{2018}           %% If different from \acmYear

%% Bibliography style
\bibliographystyle{ACM-Reference-Format}
%% Citation style
%% Note: author/year citations are required for papers published as an
%% issue of PACMPL.
\citestyle{acmauthoryear}   %% For author/year citations


%%%%%%%%%%%%%%%%%%%%%%%%%%%%%%%%%%%%%%%%%%%%%%%%%%%%%%%%%%%%%%%%%%%%%%
%% Note: Authors migrating a paper from PACMPL format to traditional
%% SIGPLAN proceedings format must update the '\documentclass' and
%% topmatter commands above; see 'acmart-sigplanproc-template.tex'.
%%%%%%%%%%%%%%%%%%%%%%%%%%%%%%%%%%%%%%%%%%%%%%%%%%%%%%%%%%%%%%%%%%%%%%


%% Some recommended packages.
\usepackage{booktabs}   %% For formal tables:
                        %% http://ctan.org/pkg/booktabs
\usepackage{subcaption} %% For complex figures with subfigures/subcaptions
                        %% http://ctan.org/pkg/subcaption


\begin{document}

%% Title information
\title[Blockchain Games]{Games Over Blockchain}         %% [Short Title] is optional;
                                        %% when present, will be used in
                                        %% header instead of Full Title.
%\titlenote{with title note}             %% \titlenote is optional;
                                        %% can be repeated if necessary;
                                        %% contents suppressed with 'anonymous'
\subtitle{Minimal Verifiable Language}                     %% \subtitle is optional
%\subtitlenote{with subtitle note}       %% \subtitlenote is optional;
                                        %% can be repeated if necessary;
                                        %% contents suppressed with 'anonymous'


%% Author information
%% Contents and number of authors suppressed with 'anonymous'.
%% Each author should be introduced by \author, followed by
%% \authornote (optional), \orcid (optional), \affiliation, and
%% \email.
%% An author may have multiple affiliations and/or emails; repeat the
%% appropriate command.
%% Many elements are not rendered, but should be provided for metadata
%% extraction tools.

%% Author with single affiliation.
\author{Elazar Gershuni}
\authornote{with author1 note}          %% \authornote is optional;
                                        %% can be repeated if necessary
\orcid{nnnn-nnnn-nnnn-nnnn}             %% \orcid is optional
\affiliation{
  \position{Student}
  \department{Computer Science}              %% \department is recommended
  \institution{Tel Aviv University}            %% \institution is required
  \streetaddress{Street1 Address1}
  \city{Tel Aviv}
  \state{State1}
  \postcode{Post-Code1}
  \country{Israel}                    %% \country is recommended
}
\email{elazarg@mail.tau.ac.il}          %% \email is recommended

\author{Noam Rinetzky}
\authornote{with author1 note}          %% \authornote is optional;
                                        %% can be repeated if necessary
\orcid{nnnn-nnnn-nnnn-nnnn}             %% \orcid is optional
\affiliation{
  \position{}
  \department{Computer Science}              %% \department is recommended
  \institution{Tel Aviv University}            %% \institution is required
  \streetaddress{Street1 Address1}
  \city{Tel Aviv}
  \state{State1}
  \postcode{Post-Code1}
  \country{Israel}                    %% \country is recommended
}
\email{maon@cs.tau.ac.il}          %% \email is recommended


%% Abstract
%% Note: \begin{abstract}...\end{abstract} environment must come
%% before \maketitle command
\begin{abstract}
Text of abstract \ldots.
\end{abstract}


%% 2012 ACM Computing Classification System (CSS) concepts
%% Generate at 'http://dl.acm.org/ccs/ccs.cfm'.
\begin{CCSXML}
<ccs2012>
<concept>
<concept_id>10011007.10011006.10011008</concept_id>
<concept_desc>Software and its engineering~General programming languages</concept_desc>
<concept_significance>500</concept_significance>
</concept>
<concept>
<concept_id>10003456.10003457.10003521.10003525</concept_id>
<concept_desc>Social and professional topics~History of programming languages</concept_desc>
<concept_significance>300</concept_significance>
</concept>
</ccs2012>
\end{CCSXML}

\ccsdesc[500]{Software and its engineering~General programming languages}
\ccsdesc[300]{Social and professional topics~History of programming languages}
%% End of generated code


%% Keywords
%% comma separated list
\keywords{keyword1, keyword2, keyword3}  %% \keywords are mandatory in final camera-ready submission


%% \maketitle
%% Note: \maketitle command must come after title commands, author
%% commands, abstract environment, Computing Classification System
%% environment and commands, and keywords command.
\maketitle


\section{Introduction}

\subsection{Motivation}
Text of paper \ldots


\subsection{Execution Model}
The execution model consists of a single stateful server, infinite number of clients, and a message-passing channel between them.
The server executes a single game described in our language. A client can send messages, and can read the state of the server and act accordingly.



\subsection{Running Example}
\subsubsection{Odds and Evens: Simultaneous Two-Player Game}

\subsubsection{Blind Auction: Simultaneous n-Player Game}

\section{The Language}

\subsection{Syntax}

\subsubsection{Multi-player Game}

\subsubsection{n-player Role: Aggregation}

\subsubsection{Combining Roles}

\subsection{Semantics}

\subsubsection{Quitting, Deadlines and the Global Clock}
Over the blockchain, we cannot avoid the possibility that any party can stop playing at any time. Additionally, players might not `live up to their promises' by failing to reveal commitments, or revealing commitments that do not hold to the contract.
Therefore, very similar to velner et al., we assume a global clock. Each step takes a single tick. If a player does not play when its turn comes, anyone can play for her, and the value of its choice is a special \texttt{None} value (thus all external choices are nullable types). The intention is that None is a sign of player deviating from the rules of the game, and such deviation should be punished. Static analysis (see below) can warn about failure to punish players for None values.

\subsection{Implementation}
\subsubsection{Translation to Solidity}


\section{Analysis}

\subsection{Hide-Reveal as Resource}
A hidden variable can be seen as a resource: it is `allocated' when sent, `released' when revealed, and in the meantime cannot be copied. Unfortunately, a very useful pattern of resource usage -- a \texttt{using} or \texttt{with} construct -- is inapropriate since the variable is not used between the time it is committed to the time it is released, but only later; additionally, variables revealed in the meantime \textit{are} used later, so tying it to a scope is not useful at all. In a sense, the situation has more in common to an uninitialized variable, except that \textit{moving} the variables' value does make sense, and immediate initialization is not useful (it is just like a using public variable).

\subsection{No Rule-breaking (Well Separation)}
The \texttt{where} clauses enables an important safety analysis: we would like to know that the opposing player cannot force us to break the rules. This question, like others, can be asked both for a well-behaved opponent and for a malicious one. For a malicious player, we want to know whether $\exists \overline{x} \forall \overline{y}: P(\overline{x})\wedge \neg Q(\overline{y})$, where $P$ and $Q$ are the formulae representing the \texttt{where} clauses of the opponent and ourselves, respectively.

For a well-behaved opponent, the analysis will use a DQBF formula similar to the one described above.

\subsection{Finding Dominant Strategy}
A dominant strategy for a player is a sequence of choices such that no matter what the opponent does, the payoff cannot be any better. Formally, given a payoff function for a player 
$F : \mathbb{N}^n \times \mathbb{N}^k \rightarrow \mathbb{N}$ 
over the game, we can look for a dominant strategy by solving
$\exists \overline{x} \forall \overline{x'} \overline{y}: F(\overline{x}, \overline{y})\geq F(\overline{x'}, \overline{y})$.

\subsubsection{Multplayer Game}

\subsubsection{N-player Game}

\subsubsection{Absence of Communication Mismatches}
We use simple session types to do sanity check on the server code, and to verify the conformance of clients to the protocol

\subsection{Equilibria}
The game is translated to an extensive form, with information sets reflecting the use of \emph{private} and \emph{publish}. 
This format can be analyzed to find Nash equilibria, taking into account the possibility of players quitting in the middle of the game.



\subsection{DQBF}
The language for full-information 2-player finite seuquential games is QBF: for example the formula $\exists x \forall y: x > y$ refers to the game where the player chooses $x$ and shows it, then the opponent chooses $y$, and the first player wins iff $x > y$. Obviously the second player has a winning strategy, e.g. always choose $y=x$.

In QBF, there is a order of quanitifiers define uniquely the information on which they depend; i.e. information dependncy is a chain. In order to model partial information, we need to be able to specify a DAG of dependencies. DQBF consists of formulae of the form $\forall x_1, x_2, \ldots x_n \exists y_1 \{\overline{z_1}\}, y_2 \{\overline{z_2}\}, \ldots y_k \{\overline{z_k}\}: P(\overline{x}, \overline{y})$ where each $z_i$ is a list of variables on which $y_i$ depends. For example, the game $\exists x \forall y\{\}: x > y$ is the symmetric game in which both players make their choices independently; in this game the second player has no winning strategy.

In our case, it doesn't make sense to represent every dependency pattern, since many patterns do not make sense in our setting: we can only hide value (from everyone else) or reveal values (to everyone). In particular, the information flow graph must be transitive.

\section{Evaluation}

\section{Related Work}

\section{Conclusion}

%% Acknowledgments
\begin{acks}                            %% acks environment is optional
                                        %% contents suppressed with 'anonymous'
  %% Commands \grantsponsor{<sponsorID>}{<name>}{<url>} and
  %% \grantnum[<url>]{<sponsorID>}{<number>} should be used to
  %% acknowledge financial support and will be used by metadata
  %% extraction tools.
  This material is based upon work supported by the
  \grantsponsor{GS100000001}{National Science
    Foundation}{http://dx.doi.org/10.13039/100000001} under Grant
  No.~\grantnum{GS100000001}{nnnnnnn} and Grant
  No.~\grantnum{GS100000001}{mmmmmmm}.  Any opinions, findings, and
  conclusions or recommendations expressed in this material are those
  of the author and do not necessarily reflect the views of the
  National Science Foundation.
\end{acks}


%% Bibliography
\bibliography{bibliography}


%% Appendix
\appendix
\section{Appendix}

Text of appendix \ldots

\end{document}
